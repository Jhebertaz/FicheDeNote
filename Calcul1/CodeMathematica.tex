\section{Code Mathematica}
\subsection{Méthode pour trouver et déterminer les extremums locaux d'une fonction à plusieurs variables}
	\begin{verbatim}
	Quit[]
	(*Trouvez les maximums locaux,les minimums locaux et
	les points de selle de la fonction.Si vous disposez d'un logiciel
	le permettant,tracez le graphe de la fonction en choisisesant un 
	domaine et un point de vue qui révèlent toutes les caractéris­
	tiques importantes de la fonction. *)
	
	submatrix[matrice_, d_] := Take[matrice[[1 ;; d, ;; d]]];
	mineur   [matrice_, d_] := Det[submatrix[matrice, d]]	
	Syvelster[matrice_]     := For[i = 1, i < Length[matrice]+1, i++,Print[Evaluate@mineur[matrice, i]]]
	
	%%%
	5.1.16
	Panel[Grid[{
	           {f := x y Exp[-(x^2 + y^2)/2]},
	           {solu = Reduce@Resolve[Grad[f, {x, y}] == 0]},
	           {mathess = D[f, {{x, y}}, {{x, y}}];},
	           {MatrixForm@mathess},
	           {ContourPlot[f, {x, -2, 2}, {y, -2, 2}], 
	                      Plot3D[f, {x, -2, 2}, {y, -2, 2}]},
	           {{x, y} = {-1, 1};},
	           {Syvelster[mathess]}}]]
	%%%
	5.1.20
	Panel[Grid[{
	           {f := Sin[x] Sin[y]},
	           {solu =	Reduce@Resolve[Grad[f, {x, y}] == 0,
	           			{x, y} \[Element] ImplicitRegion[-Pi <= x <= Pi && -Pi <= y <= Pi , {x, y}]]},
	           {mathess = D[f, {{x, y}}, {{x, y}}];},
	           {MatrixForm@mathess},
	           {ContourPlot[Sin[x] Sin[y], {x, -Pi, Pi}, {y, -Pi, Pi}], 
	                      Plot3D[Sin[x] Sin[y], {x, -Pi, Pi}, {y, -Pi, Pi}]},
	           {{x, y} = {0, 0};},
	           {Syvelster[mathess]}}]]
	\end{verbatim}
	\begin{center}***\end{center}
\subsection{Méthode des multiplicateurs de Lagrange pour trouver les extremums sous contraintes}
		\begin{verbatim}
		(*Utilisez les multiplicateurs de Lagrange pour trouver le maxi­
		mum et le minimum de la fonction sous les contraintes données.*)
		%%%
		5.3.8
		Panel[Grid[{
		           {f[x_, y_] := x Exp[y], g[x_, y_] := x^2 + y^2},
		           {solu = Solve[Grad[f[x, y], {x, y}] == mu Grad[g[x, y], 
		           {x, y}] && g[x, y] == 2, {x, y, mu}, Reals]}}]]
		           {f[1, 1], f[-1, 1]}
	\end{verbatim}
	\begin{center}***\end{center}
	\begin{verbatim}
		%%%
		5.3.14
		Panel[Grid[{
		           {f[x_, y_, z_] := x^4 + y^4 + z^4, g[x_, y_, z_] := x^2 + y^2 + z^2},
		           {solu = Solve[Grad[f[x, y, z], {x, y, z}] == mu Grad[g[x, y, z], 
		           {x, y, z}] && g[x, y, z] == 1, {x, y, z, mu}, Reals]},
		           {Minimize[{f[x, y, z], g[x, y, z] == 1}, {x, y, z}]},
		           {Maximize[{f[x, y, z], g[x, y, z] == 1}, {x, y, z}]}
		}]]
	\end{verbatim}
	\begin{center}***\end{center}
	\begin{verbatim}
		%%%
		5.3.18
		Panel[Grid[{
		           {f[x_, y_, z_] := x^2 + y^2 + z^2, g[x_, y_,z_] := x-y,	h[x_, y_, z_] := y^2 - z^2},
		           {solu = Solve[Grad[f[x, y, z], {x, y, z}] == mu Grad[g[x, y, z], {x, y, z}] + 
		                nu Grad[h[x, y, z], {x, y, z}] && g[x, y, z] == 1 && 
		                    h[x, y, z] == 1, {x, y, z, mu, nu}, Reals]},
		           {Minimize[{f[x, y, z], g[x, y, z] == 1 && h[x, y, z] == 1}, {x, y, z}]},
		           {Maximize[{f[x, y, z], g[x, y, z] == 1 && h[x, y, z] == 1}, {x, y,z}]}
		}]]
	\end{verbatim}
	Mathematica indique qu'aucun maximum est atteint, mais le maximum est atteint au point (2,1,0).	
	\begin{center}***\end{center}

\subsection{Méthode pour trouver les extremums globaux sous une contrainte d'inégalité}
	\begin{verbatim}
		(*Trouvez les extremums globaux de f, s'ils existent,sur la
		région décrite par l'inégalité.*)
		%%%
		5.3.38
		Panel[Grid[{
		           {f[x_, y_, z_] := z Exp[x y], g[x_, y_, z_] := x^2 + y^2 + z^2},
		           {Minimize[{f[x, y, z], g[x, y, z] <= 9}, {x, y, z}]},
		           {Maximize[{f[x, y, z], g[x, y, z] <= 9}, {x, y, z}]}
		}]]
	\end{verbatim}
	\begin{center}***\end{center}
	\subsection{Intégrale double}
	\begin{verbatim}
	
	(*Calculez l'intégrale double en la considérant comme le volume d'un solide.*)
	%%%
	6.1.9
	Panel[Grid[{
	           {Integrate[Sqrt[2], {x, 2, 6}, {y, -1, 5}]},
	           {Integrate[Sqrt[2], {x, y} \[Element] ImplicitRegion[2 <= x <= 6 && -1 <= y <= 5, {x, y}]]}
	}]]
	\end{verbatim}
	\begin{center}***\end{center}
	\begin{verbatim}
	(*Calculez l'intégrale double.*)
	%%%
	6.2.9
	Panel[Grid[{
	           {f := Exp[-y^2]},
	           {Integrate[f, {y, 0, 3}, {x, 0, y}]},
	           {Integrate[f, {x, y} \[Element] ImplicitRegion[0 <= y <= 3 && 0 <= x <= y, {x, y}]]}
	}]]

	%%%
	6.2.17
	Panel[Grid[{
	           {f := 2 x - y},
	           {Integrate[f, {x, y} \[Element] Circle[{0, 0}, 2]]}
	}]]
	
	%%%
	6.2.25
	Panel[Grid[{
	           {f := 1},
	           {Integrate[f, {x, y, z} \[Element] ImplicitRegion[0 <= z <= x^2 && x^2 <= y <= 4, {x, y, z}]]}
	}]]
	\end{verbatim}
	\begin{center}***\end{center}
\subsection{Intégrale triple}
	\begin{verbatim}
	Exemple 7.3.2
	(*Calcul d'intégrale triple*)
	Panel[Grid[{
	           {R = ImplicitRegion[Sqrt[x^2 + y^2] <= z <= 2, {x, y, z}]},
	           {"Result :", Integrate[(x^2 + y^2), {x, y, z} \[Element] R]},
	           {DiscretizeRegion[R]},
	           {ClearAll["Global`"]}}]]
	\end{verbatim}
	\begin{center}***\end{center}
	\begin{verbatim}
	%%%
	7.4.2
	(*Calcul de volume*)
	Panel[Grid[{
	           {"Result :", Integrate[\[Rho]^2 Sin[\[Phi]], {\[Rho], 0, 3},
	           {\[Theta], 0,Pi/2} , { \[Phi], 0, Pi/6}]}
	}]]
	\end{verbatim}
	\begin{center}***\end{center}
	\begin{verbatim}
	%%%
	7.4.5
	(*Calcul d'intégrale triple*)
	Panel[Grid[{
	           {"Region :", region745 = ImplicitRegion[x^2 + y^2 + z^2 <= 25, {x, y, z}]},
	           {"Result :", Integrate[(x^2 + y^2 + z^2)^2, {x, y, z} \[Element] region745]}
	}]]
	
	%%%
	7.4.8
	Panel[Grid[{
	           {"Region :",region748 = ImplicitRegion[x^2 + y^2 + z^2 <= 9 && y >= 0, {x, y, z}]},
	           {"Result :", Integrate[y^2, {x, y, z} \[Element] region748]}
	}]]
	
	%%%
	7.4.26
	(*Calcul de volume borné par des surfaces*)

	\
	Panel[Grid[{
	           {"Region :", region7426 = ImplicitRegion[x^2 + y^2 + (z - 1)^2 <= 16
	            && x^2 + y^2 >= 4, {x, y, z}]},
	           {"Result :", Integrate[1, {x, y, z} \[Element] region7426]},
	           {DiscretizeRegion[region7426]}}]]
	\end{verbatim}
	\begin{center}***\end{center}	
\subsection{Méthode pour calculer des volumes bornés par des surfaces}
		\begin{flushleft}
			\verb|R = ImplicitRegion[x x + y y <= z && x x + y y <= 25 && z <= 25, {x, y, z}]|\\
			\verb|RegionBoundary[R]|\\
			\verb|DiscretizeRegion[R, MeshQualityGoal -> "Maximal"]|\\
			\verb|Volume[R]|\\
			\verb|RegionMeasure[R, 3]|
		\end{flushleft}	