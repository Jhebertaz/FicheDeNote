\section{Code Mathematica}
	\begin{center}Méthode pour trouver et déterminer les extremums locaux d'une fonction à plusieurs variables.\end{center}
	
	\begin{verbatim}
	\Quit[]
	(*Trouvez les maximums locaux,les minimums locaux et
	les points de selle de la fonction. Si vous disposez d'un logiciel
	le permettant,tracez le graphe de la fonction en choisissant un 
	domaine et un point de vue qui révèlent toutes les caractéris­
	tiques importantes de la fonction. *)
	
	submatrix[matrice_, d_] := Take[matrice[[1 ;; d, ;; d]]];
	mineur[matrice_, d_] := Det[submatrix[matrice, d]]
	(*evalSyvelster[matrice_]:=For[i=1,i<Length[matrice]+1,i++,
	If[mineur[matrice,i]<0,Print["-"],If[mineur[matrice,i]>0,
	Print["+"],Print["0"]]]]*)
	
	evalSyvelster[matrice_] := 
	        For[i = 1, i < Length[matrice] + 1, i++, 
	           Print[Evaluate@mineur[matrice, i]]]
	5.1.16
	Panel[Grid[{
	    {f := x y Exp[-(x^2 + y^2)/2]},
	    {solu = Reduce@Resolve[Grad[f, {x, y}] == 0]},
	    {mathess = D[f, {{x, y}}, {{x, y}}];},
	    {MatrixForm@mathess},
	    {{x, y} = {-1, 1};},
	    {evalSyvelster[mathess]}}]]
	\end{verbatim}
	\begin{center}***\end{center}
\begin{center}Méthode pour calculer des volumes bornée par des surfaces \end{center}
		\begin{flushleft}
			\verb|R = ImplicitRegion[x x + y y <= z && x x + y y <= 25 && z <= 25, {x, y, z}]|\\
			\verb|RegionBoundary[R]|\\
			\verb|DiscretizeRegion[R, MeshQualityGoal -> "Maximal"]|\\
			\verb|Volume[R]|\\
			\verb|RegionMeasure[R, 3]|
		\end{flushleft}	