\documentclass{article}[babel]
	\addtolength{\oddsidemargin}{-.875in}
	\addtolength{\evensidemargin}{-.875in}
	\addtolength{\textwidth}{1.75in}
	\addtolength{\topmargin}{-.875in}
	\addtolength{\textheight}{1.75in}
	\usepackage{fancyhdr}
	\usepackage{hyperref}
	\usepackage{amsmath,amssymb,amsthm,etoolbox,mathtools,verbatim,enumitem,imakeidx}
	\usepackage{caption,pdfpages,graphicx}
	\DeclareGraphicsExtensions{.pdf,.png}
	\DeclarePairedDelimiter\ceil{\lceil}{\rceil}
	\DeclarePairedDelimiter\floor{\lfloor}{\rfloor}
	\pagestyle{fancy}
	\lhead{Julien Hébert-Doutreloux}
	\rhead{--Page \thepage}
	\cfoot{Calcul I}
	\renewcommand{\headrulewidth}{0.4pt}
	\renewcommand{\footrulewidth}{0.4pt}
	\newtheorem{mydef}{Définition}
	\newtheorem{myprop}{Proposition}
	\newtheorem{mythm}{Théorème}
	\indexsetup{othercode=\small}
	\makeindex[program=makeindex,columns=2,intoc=true,options={-s index_style.ist}]
	\newcounter{code}
	\newenvironment{code}[1]
	{\index{#1}\refstepcounter{code}\rule{1ex}{1ex}\hspace{\stretch{1}} }
	{ \hspace{\stretch{1}}\rule{1ex}{1ex} }
\begin{document}
	\begin{titlepage}
		\centering
		\includegraphics[width=0.5\textwidth]{Universite_de_Montreal_logo}\par\vspace{1cm}
		{\scshape\LARGE Université de Montréal\par}
		\vspace{1cm}
		{\scshape\Large Fiche Récapitulative\par}
		\vspace{1.5cm}
		{\huge\bfseries Calcul I\par}
		\vspace{2cm}
		{\Large\itshape Julien Hébert-Doutreloux\par}
		\vfill
		%supervised by\par
		%Dr.~Mark \textsc{Brown}
		\vfill
		% Bottom of the page
		{\large \today\par}
	\end{titlepage}
		\tableofcontents
		\newpage
\section{Les dérivées des fonctions de plusieurs variables}
	\subsection{Les dérivées partielles}
		\begin{mythm}\index{Théorème de Clairaut}[Clairaut]
			Soit une fonction $f$ définie sur un disque $D$ qui contient le point $(a,b)$. Si les fonction $f_{xy}$ et $f_{yx}$ sont continues sur $D$, alors \[f_{xy}(a,b)=f_{yx}(a,b)\]
		\end{mythm}
		\begin{code}{Derivative}
		Gives the multiple partial derivative \verb|D[f,{x,n},{y,m},...]|
	\end{code}
	\subsection{Les plans tangents et approximations linéaires}
		\subsubsection*{Les plans tangents}
			\begin{mydef}
					Si $f$ possède des dérivées partielles continues, alors l'équation du plan tangent à la surface $z=f(x,y)$ au point $P(x_0,y_0,z_0)$ est \[z-z_0=f_x(x_0,y_0)(x-x_0)+f_y(x_0,y_0)(y-y_0).\]
			\end{mydef}
		\subsubsection*{Les approximations linéaires}
			\begin{mydef}
				La fonction linéaire dont le graph est ce plan tangent, à savoir \[L(x,y)=f(a,b)+f_x(a,b)(x-a)+f_y(a,b)(y-b)\]
				est appelée \textbf{linéarisation} de $f$ en $\vec{c}=(c_1,c_2,...,c_n)$, et l'approximation
				\[f(\vec{x})\approxeq f(\vec{c})+\sum_{i=1}^n f_{x_i}(\vec{c})(x_i-c_i)\]
				est appelée \textbf{approximation linéaire} de $f$ en $\vec{c}$. La différentielle est \[dw=\sum_{i=1}^{n}\frac{\partial w}{\partial x_i}\,dx_i\]
			\end{mydef}
			\begin{mythm}
				Si les dérivées partielles $f_x$ et $f_y$ existent près de $(a,b)$ et sont continues en $(a,b)$, alors $f$ est différentiable en $(a,b)$.
			\end{mythm}
	\subsection{La règle de dérivation en chaîne}
		\begin{mythm}[Règle de dérivation en chaîne]
			Si $u$ est une fonction différentiable de $n$ variables $x_1,x_2,...,x_n$, et si chaque $x_j$ est une fonction différentiable des $m$ variables $t_1,t_2,...,t_m$, alors $u$ est une foncition différentiable de $t_1,t_2,...,t_m$ et
			\[\frac{\partial u}{\partial t_i}=\sum_{i=1}^{n}\frac{\partial u}{\partial x_i}\frac{\partial x_i}{\partial t_i}\] 
		\end{mythm}
	\subsection{Les dérivées directionnelles et le vecteur gradient}
		\subsubsection*{Les dérivées directionnelles}
			\begin{mydef}\index{Dérivée directionnelle}
				La \textbf{dérivée directionnelle} (si la limite existe) de $f$ dans la direction d'un vecteur unitaire $\vec{u}=(a,b,c)$ en $(x_0,y_0,z_0)$ est \[f_{\vec{u}}(x_0,y_0,z_0)=\lim_{h\to 0}\frac{f(x_0+ha,y_0+hb,z_0+hc)-f(x_0,y_0,z_0)}{h}\]
			\end{mydef}
			\begin{mydef}\index{Vecteur gradient}
				Le \textbf{vecteur gradient}, noté $\nabla f$ ou \normalfont{grad} $f$, d'une fonction à $n$ variables est \[\Big(\frac{\partial f}{\partial x_1},\frac{\partial f}{\partial x_2},...,\frac{\partial f}{\partial x_n}\Big)\]
				D'où la dérivée dans la direction $\vec{u}$, un vecteur unitaire, est \[f_{\vec{u}}(\vec{x})=\nabla f(\vec{x})\bullet \vec{u}\]
			\end{mydef}
			\begin{code}{Gradient}
			Gradient of a scalar function $(\partial f / \partial x_1,...,\partial f / \partial _n)$ : \verb|Grad[f,{x_1,x_2,...,x_n}]| ou \verb|D[f[x, y,...], {{x, y,...}}]|
		\end{code}
		\subsubsection*{Les plans tangents aux surfaces de niveau}
			\begin{mydef}
				Soit une surface $S$ d'équation $F(x,y,z)=k$ et $P(x_0,y_0,z_0)$, un point de $S$. Si $\nabla F(P)\neq\vec{O}$, alors le \textbf{plan tangent à la surface de niveau} $F(x,y,z)=k$ en $P$ est \[\nabla F(P) \bullet (\vec{x}-P)=0\]
			\end{mydef}
			\begin{myprop}
				Soit $f$ une fonction différentiable de $n$ variables, et $\vec{x}$ un point de $\mathbb{R}^n$. Alors,
				\begin{itemize}
					\item la dérivée directionnelle de $f$ en $\vec{x}$ est maximale dans la direction du gradient $\nabla f(\vec{x})$
					\item la taux de variation maximal de $f$ en $\vec{x}$ est $||\nabla f(\vec{x})||$
					\item si $\vec{u}\perp \nabla f(\vec{x})$, alors la $ \nabla f(\vec{x})\bullet\vec{u}=0$
					\item le gradient $\nabla f(\vec{x})$ est perpendiculaire à l'ensemble de niveau de $f$ passant par $\vec{x}$
				\end{itemize}
			\end{myprop}
	\subsection{Les approximations de Taylor en deux variables}
		\begin{mydef}\index{Polynôme de Taylor}[Polynôme de Taylor de degré $1$ de $f$ en $(a,b)$]
			\[L(x,y)=f(a,b)+f_x(a,b)(x-a)+f_y(a,b)(y-b)\]
		\end{mydef}
		\begin{mydef}\index{Polynôme de Taylor}[Polynôme de Taylor de degré $2$ de $f$ en $(a,b)$]
			\[Q(x,y)=L(x,y) +\frac{1}{2!}f_xx(a,b)(x-a)^2+f_xy(a,b)(x-a)(y-b)+\frac{1}{2!}f_yy(a,b)(y-b)^2\]
		\end{mydef}
\section{L'optimisation}
	\subsection{Les valeurs extrêmes des fonctions de deux variables}
		\begin{mydef}\index{Matrice Hessienne}(Matrice Hessienne)
			La matrice Hessienne d'une fonction $f$ à $n$ variables est la matrice carrée d'ordre $n$ notée $\nabla^2 f$ telle que l'élément $(\nabla^2 f)_{ij} = f_{x_i x_j}$ : 
			\[\nabla^2 f=\begin{bmatrix}
			f_{x_1 x_1} &f_{x_1 x_2} & ... &f_{x_1 x_n}\\
			f_{x_2 x_1} &f_{x_2 x_2} & ... &f_{x_2 x_n}\\
			\vdots &\vdots & \ddots &\vdots\\
			f_{x_n x_1} &f_{x_n x_2} & ... &f_{x_n x_n}
			\end{bmatrix}\]
		\end{mydef}	
		\begin{code}{Matrice Hessienne}
			Le matrice Hessienne : \verb|MatrixForm[D[f[x, y,...], {{x, y,...}}, {{x, y,...}}]]| 
		\end{code}
		\begin{mydef}\index{Point critique}[Point critique]
			Soit $f$ une fonction à $n$ variables et $P=(c_1,c_2,...,c_n)$, un point. Le point $P$ est un point critique si $\nabla f (P)=\vec{O}$.
		\end{mydef}
		\begin{mythm}\index{Test des dérivées premières}[Test des dérivées premières]
			Si $f$ possède un maximum (resp. minimum) local en $(a,b)$ et si les dérivées partielles du premiers ordre de $f$ existent, alors $\nabla f(a,b)=\vec{O}.$
		\end{mythm}
		\subsubsection*{Les maximums et minimum absolus}
			\begin{mythm}\index{Bornes atteintes}[Bornes atteintes]
				Si $f$ est continu sur un compact $K$, alors $f$ atteint sont maximum (resp. sont minimum) absolus en au moins un points de $K$. Autrement dit, \[\exists \vec{x}_1,\vec{x}_1\in K : f(\vec{x}_1)=\inf_{\vec{x}\in K}\{f(\vec{x})\}\quad\text{et}\quad f(\vec{x})_2=\sup_{\vec{x}\in K}\{f(\vec{x})\}\]
			\end{mythm}
	\subsection{L'optimisation des fonctions de plusieurs variables}
		\begin{mydef}
			Un point critique $\vec{a}$ est un point de selle de la fonction $f$si, dans toute boule ouverte $B_{\varepsilon}(\vec{a})$, il existe des points $\vec{x}_1$ et $\vec{x}_2$ tels que $f(\vec{x}_1)<f(x)<f(\vec{x}_2)$.
		\end{mydef}
		\subsubsection*{Le signe d'une matrice}
			\begin{mythm}\index{Critère de Sylvester}[Critère de Sylvester]
				Soit $A$ une matrice symétrique inversible 
				\begin{itemize}
					\item Si $\alpha_j>0$ pour $j=1,2,...,n$, alors $A$ est définie positive.
					\item Si $\beta_j>0$ pour $j=1,2,...,n$, alors $A$ est définie négative.
				\end{itemize}
			\end{mythm}
			\begin{code}{Sylvester}
				Calculer les mineurs principaux d'une matrice\hfill : 
				\begin{flushleft}
				\verb|submatrix[matrice_, d_] := Take[matrice[[1 ;; d, 1 ;; d]]]|\\
			\verb|mineur[matrice_, d_] := Det[submatrix[matrice, d]]|\\
				\verb|{x, y, z} = {0, 1, 2}|\\
				\verb|For[i = 1, i < 4, i++, Print[mineur[matrice, i]]]|
				\end{flushleft}
				\indent\indent\indent\indent où \verb|d| le numéro de la colonne du ième élément de la diagonale principale. 
			\end{code}
		\begin{mythm}{Conditions suffisantes du deuxièmes ordre pour un problème d'optimisation sans contraintes}
			\begin{itemize}
				\item si $\nabla^2 f(\vec{a})$ est définie positive (resp. négative), alors $f$ possède un minimum (resp. un maximum) local en $\vec{a}$.
				\item si $\nabla^2 f(\vec{a})$ est indéfiniem alors $\vec{a}$ est un point de selle de $f$
			\end{itemize}
		\end{mythm}
	\begin{code}{Optimization avec contrainte}
		Exemple de résolution pour trouver les points critique\hfill ;
		\begin{flushleft}
			\verb|F[x_, y_, z_] := x^3 - x y + y^2 + z^2|
			\verb|gradient = Grad[F[x, y, z], {x, y, z}]|
			\verb|Solve[Resolve[{gradient == mu Grad[x x + y y + z z, {x, y, z}] &&|\\\verb| x x + y y + z z == 1}, {mu, x, y, z}, Reals]]|
		\end{flushleft}
	\end{code}

\section{Les intégrales doubles}
	\begin{mydef}\index{Volume}
	Si $f(x,y)\geq 0$, alors le volume $V$ du solide au-dessus du rectangle $R$ et sous la surface $z=f(x,y)$ est \[V=\iint\limits_R f(x,y)\, dA\]
\end{mydef}
	\begin{mydef}\index{Intégrales itérées}
		Selon l'ordre d'intégration, les intégrales itérées sont ;
		\begin{gather}
			\int_a^b\int_c^d f(x,y)\, dy\, dx\\
			\int_c^d\int_a^b f(x,y)\, dx\, dy
		\end{gather}
	\end{mydef}
	\begin{mythm}\index{Théorème de Fubini}
		Si $f$ est continue sur le rectangle $R=\big\{(x,y) \lvert a\leq x \leq b, c\leq y \leq d  \big\}$, alors 
			\[\iint\limits_R f(x,y)\, dA=\int_a^b\int_c^d f(x,y)\, dy\, dx=	\int_c^d\int_a^b f(x,y)\, dx\, dy\]
		Si on suppose que $f$ est bornée sur $R$, que $f$ est discontinue sur un nombre fini de courbes lisses et que les intégrales itérées existent.
	\end{mythm}
	\subsection{Les intégrales doubles sur des domaines généraux}
		\begin{mydef}\index{Domaine de type I}
			Si $f$ est continue sur un région $D$ de type I,
				\[D=\big\{(x,y)\lvert a\leq x\leq b, g_1(x)\leq y\leq g_2(x)\big\}\]
				\[\iint\limits_D f(x,y)\, dA = \int_a^b\int_{g_1(x)}^{g_2(x)} f(x,y)\, dy\, dx\]
		\end{mydef}
		\begin{mydef}\index{Domaine de type II}
			Si $f$ est continue une région $D$ de type II,
				\[D=\big\{(x,y)\lvert c\leq y\leq c, h_1(y)\leq x\leq h_2(y)\big\}\]
			\[\iint\limits_D f(x,y)\, dA = \int_c^d\int_{h_1(y)}^{h_2(y)} f(x,y)\, dx\, dy\]
		\end{mydef}
		\begin{myprop}\index{Aire}\index{Volume}
		Si on intègre la fonction constante $f(x,y)=1$ sur une région $D$, on obtient l'aire de $D$, car le volume sous $f(x,y)=1$ au-dessus de $D$ est égal à l'aire de $D$;\[\iint\limits_D 1\, dA=A(D)\]
	\end{myprop}
	\subsection{Système de coordonnée}\index{Système de coordonnée}
		\begin{mydef}\index{Polaire}
			Le passage du système cartésien au système polaire est donné par les relations suivantes;
				\[r^2=x^2+y^2\qquad x=r\cos\theta\qquad y=r\sin\theta\]
		\end{mydef}
		\begin{mydef}\index{Cylindrique}
			Le passage du système cartésien au système cylindrique est donné par les relations suivantes;
			\[r^2=x^2+y^2\qquad x=r\cos\theta\qquad y=r\sin\theta\qquad z=z\]
		\end{mydef}
		\begin{mydef}\index{Sphérique}
	Le passage du système cartésien au système sphérique est donné par les relations suivantes;
	\[\rho^2=x^2+y^2+z^2\qquad x=\rho\sin\phi\cos\theta\qquad y=\rho\sin\phi\sin\theta\qquad z=\rho\cos\phi\]
\end{mydef}
	\subsection{Les intégrales doubles en coordonnées polaires}
		\begin{mydef}\index{Intégrales doubles en polaire}
		Si $f$ est continue sur un rectangle polaire $R$ défini par $0\leq a\leq r\leq b,\alpha\leq\theta\leq\beta$, où $0\leq\beta -\alpha \leq 2\pi$, alors \[\iint\limits_R f(x,y)\, dA=\int_{\alpha}^{\beta}\int_a^b f(r\cos\theta,r\sin\theta)r\,dr\, d\theta\]
		Si $f$ est continue sur région polaire de la forme \[D=\big\{ (r,\theta)\lvert \alpha\leq\theta\leq\beta, h_1(\theta)\leq r \leq h_2(\theta) \big\}\]
		\[\iint\limits_D f(x,y)\, dA = \int_{\alpha}^{\beta}\int_{h_1(\theta)}^{h_2(\theta)} f(r\cos\theta,r\sin\theta)r\, dr\, d \theta\]
	\end{mydef}
\section{Les intégrales triples}
	\subsection{Les intégrales triples}
		\begin{mythm}\index{Théorème de Fubini pour les intégrales triples}
			Si $f$ est contniue sur le rectangle $B=[a,b]\times[c,d]\times[r,s]$, alors
			\[\iiint\limits_B f(x,y,z)\, dV=\int_r^s\int_c^d\int_a^b f(x,y,z)\, dx\, dy\, dz\]
		\end{mythm}
		\begin{mydef}\index{Domaine de type 1}
			Une région solide $E$ est dite de type 1 si elle est située entre les graphes de deux fonctions continues de $x$ et $y$,
			\[E=\big\{(x,y,z) \vert (x,y)\in D, u_1(x,y)\leq z\leq u_2(x,y) \big\}\]
			\[\iiint\limits_E f(x,y,z)\, dV = \iint\limits_D\bigg[\int_{u_1(x,y)}^{u_2(x,y)}f(x,y,z)\, dz\bigg]\, dA\]
		\end{mydef}
		\begin{mydef}\index{Domaine de type 1.I}
			Si la projection $D$ de $E$ dans le plan $xy$ est une région de type I, alors
			\[E=\big\{(x,y,z) \vert a\leq x\leq b,g_1(x)\leq y\leq g_2(x), u_1(x,y)\leq z\leq u_2(x,y) \big\}\]
			\[\iiint\limits_E f(x,y,z)\, dV = \int_a^b\int_{g_1(x)}^{g_2(x)}\int_{u_1(x,y)}^{u_2(x,y)}f(x,y,z)\, dz\, dy\, dx\]
		\end{mydef}
		\begin{mydef}\index{Domaine de type 1.II}
			Si  $D$ est une région de type II, alors
			\[E=\big\{(x,y,z) \vert c\leq y \leq d,h_1(y)\leq x\leq h_2(y), u_1(x,y)\leq z\leq u_2(x,y) \big\}\]
			\[\iiint\limits_E f(x,y,z)\, dV = \int_c^d\int_{h_1(y)}^{h_2(y)}\int_{u_1(x,y)}^{u_2(x,y)}f(x,y,z)\, dz\, dx\, dy\]
		\end{mydef}
		\begin{mydef}\index{Domaine de type 2}
			Une région solide $E$ est de type 2 si elle est de la forme
			\[E=\big\{(x,y,z) \vert  (y,z)\in D, u_1(y,z)\leq z\leq u_2(y,z) \big\}\]
			\[\iiint\limits_E f(x,y,z)\, dV = \iint\limits_D\bigg[ \int_{u_1(y,z)}^{u_2(y,z)}f(x,y,z)\, dx\bigg]\, dA\]
		\end{mydef}
		\begin{mydef}\index{Domaine de type 3}
			Une région solide $E$ est de type 3 si elle est de la forme
			\[E=\big\{(x,y,z) \vert  (x,z)\in D, u_1(x,z)\leq z\leq u_2(x,z) \big\}\]
			\[\iiint\limits_E f(x,y,z)\, dV = \iint\limits_D\bigg[ \int_{u_1(x,z)}^{u_2(x,z)}f(x,y,z)\, dy\bigg]\, dA\]
		\end{mydef}
		\begin{myprop}\index{Volume}
		Soit la fonction $f(x,y,z)=1$ pour tout points de $E$, alors l'intégrale triple représente le volume de $E$, \[V(E)=\iiint\limits_E\, dV\]
	\end{myprop}
	\subsection{Les intégrales triples en coordonnées cylindriques}
		\begin{mydef}\index{Intégrales triples en cylindrique}
		\[\iiint\limits_E f(x,y,z)\, dV =\int_{\alpha}^{\beta}\int_{h_1(\theta)}^{h_2(\theta)}\int_{u_1(r\cos\theta,r\sin\theta)}^{u_2(r\cos\theta,r\sin\theta)}f(r\cos\theta,r\sin\theta,z)r\, dx\, dr\, d\theta\]
	\end{mydef}
	\subsection{Les intégrales triples en coordonnées sphériques}
		\begin{mydef}\index{Intégrales triples en sphérique}
	\[\iiint\limits_E f(x,y,z)\, dV=\int_c^d\int_{\alpha}^{\beta}\int_a^b f(\rho\sin\phi\cos\theta,\rho\sin\phi\sin\theta,\rho\cos\phi)\rho^2\sin\phi\, d\rho\, d\theta\, d\phi \] où $E$ est un coin sphérique défini par
	\[E=\big\{ (\rho,\theta,\phi) \lvert a\leq \rho \leq b, \alpha\leq \theta\leq \beta, c\leq\phi\leq d\big\}\]
	Aussi,
	\[\iiint\limits_E f(x,y,z)\, dV=\int_c^d\int_{\alpha}^{\beta}\int_{g_1(\theta,\phi)}^{g_2(\theta,\phi)} f(\rho\sin\phi\cos\theta,\rho\sin\phi\sin\theta,\rho\cos\phi)\rho^2\sin\phi\, d\rho\, d\theta\, d\phi \] où $E$ est un coin sphérique défini par
	\[E=\big\{ (\rho,\theta,\phi) \lvert \alpha\leq\theta\leq\beta, c\leq\phi\leq d, g_1(\theta,\phi)\leq\rho\leq g_2(\theta,\phi) \big\}\]
	
\end{mydef}
\section{Code Mathematica}
\begin{code}{Volume}
	Méthode pour calculer des volumes bornée par des régions\hfill ;
	\begin{flushleft}
	\verb|R = ImplicitRegion[x x + y y <= z && x x + y y <= 25 && z <= 25, {x, y, z}]|\\
	\verb|RegionBoundary[R]|\\
	\verb|DiscretizeRegion[R, MeshQualityGoal -> "Maximal"]|\\
	\verb|Volume[R]|\\
	\verb|RegionMeasure[R, 3]|
	\end{flushleft}
\end{code}
		\newpage
		\printindex
\end{document}