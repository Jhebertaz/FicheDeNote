\section{Les intégrales doubles}
	\begin{mydef}\index{Volume}
	Si $f(x,y)\geq 0$, alors le volume $V$ du solide au-dessus du rectangle $R$ et sous la surface $z=f(x,y)$ est \[V=\iint\limits_R f(x,y)\, dA\]
\end{mydef}
	\begin{mydef}\index{Intégrales itérées}
		Selon l'ordre d'intégration, les intégrales itérées sont ;
		\begin{gather}
			\int_a^b\int_c^d f(x,y)\, dy\, dx\\
			\int_c^d\int_a^b f(x,y)\, dx\, dy
		\end{gather}
	\end{mydef}
	\begin{mythm}\index{Théorème de Fubini}
		Si $f$ est continue sur le rectangle $R=\big\{(x,y) \lvert a\leq x \leq b, c\leq y \leq d  \big\}$, alors 
			\[\iint\limits_R f(x,y)\, dA=\int_a^b\int_c^d f(x,y)\, dy\, dx=	\int_c^d\int_a^b f(x,y)\, dx\, dy\]
		Si on suppose que $f$ est bornée sur $R$, que $f$ est discontinue sur un nombre fini de courbes lisses et que les intégrales itérées existent.
	\end{mythm}
	\subsection{Les intégrales doubles sur des domaines généraux}
		\begin{mydef}\index{Domaine de type I}
			Si $f$ est continue sur un région $D$ de type I,
				\[D=\big\{(x,y)\lvert a\leq x\leq b, g_1(x)\leq y\leq g_2(x)\big\}\]
				\[\iint\limits_D f(x,y)\, dA = \int_a^b\int_{g_1(x)}^{g_2(x)} f(x,y)\, dy\, dx\]
		\end{mydef}
		\begin{mydef}\index{Domaine de type II}
			Si $f$ est continue une région $D$ de type II,
				\[D=\big\{(x,y)\lvert c\leq y\leq c, h_1(y)\leq x\leq h_2(y)\big\}\]
			\[\iint\limits_D f(x,y)\, dA = \int_c^d\int_{h_1(y)}^{h_2(y)} f(x,y)\, dx\, dy\]
		\end{mydef}
		\begin{myprop}\index{Aire}\index{Volume}
		Si on intègre la fonction constante $f(x,y)=1$ sur une région $D$, on obtient l'aire de $D$, car le volume sous $f(x,y)=1$ au-dessus de $D$ est égal à l'aire de $D$;\[\iint\limits_D 1\, dA=A(D)\]
	\end{myprop}
	\subsection{Système de coordonnée}\index{Système de coordonnée}
		\begin{mydef}\index{Polaire}
			Le passage du système cartésien au système polaire est donné par les relations suivantes;
				\[r^2=x^2+y^2\qquad x=r\cos\theta\qquad y=r\sin\theta\]
		\end{mydef}
		\begin{mydef}\index{Cylindrique}
			Le passage du système cartésien au système cylindrique est donné par les relations suivantes;
			\[r^2=x^2+y^2\qquad x=r\cos\theta\qquad y=r\sin\theta\qquad z=z\]
		\end{mydef}
		\begin{mydef}\index{Sphérique}
	Le passage du système cartésien au système sphérique est donné par les relations suivantes;
	\[\rho^2=x^2+y^2+z^2\qquad x=\rho\sin\phi\cos\theta\qquad y=\rho\sin\phi\sin\theta\qquad z=\rho\cos\phi\]
\end{mydef}
	\subsection{Les intégrales doubles en coordonnées polaires}
		\begin{mydef}\index{Intégrales doubles en polaire}
		Si $f$ est continue sur un rectangle polaire $R$ défini par $0\leq a\leq r\leq b,\alpha\leq\theta\leq\beta$, où $0\leq\beta -\alpha \leq 2\pi$, alors \[\iint\limits_R f(x,y)\, dA=\int_{\alpha}^{\beta}\int_a^b f(r\cos\theta,r\sin\theta)r\,dr\, d\theta\]
		Si $f$ est continue sur région polaire de la forme \[D=\big\{ (r,\theta)\lvert \alpha\leq\theta\leq\beta, h_1(\theta)\leq r \leq h_2(\theta) \big\}\]
		\[\iint\limits_D f(x,y)\, dA = \int_{\alpha}^{\beta}\int_{h_1(\theta)}^{h_2(\theta)} f(r\cos\theta,r\sin\theta)r\, dr\, d \theta\]
	\end{mydef}
\section{Les intégrales triples}