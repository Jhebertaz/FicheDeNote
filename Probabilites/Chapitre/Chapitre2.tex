%prop: proposition
%def : definition
%eq  : equation
%id	 : identity
%\section[Chapitre 2]{Axiomes de probabilités}
%	\begin{mydef}[Ensemble fondamental]\index{Ensemble fondamental}
%		L'ensemble des résultats possibles d'une expérience est appelé ensemble fondamental et est noté $S$.
%	\end{mydef}%
%	\begin{mydef}[Événement]\index{Événement}
%		Tout sous-ensemble $E$ de $S$ est appelé événement.
%		\begin{remark}~
%			\begin{enumerate}
%				\item L'événement $\{a\}$ contenant un seul élément de $S$ est appelé événement élémentaire.\index{Événement élémentaire}
%				\item L'ensemble vide noté $\O$ et $S$ sont des événements. Le premier est appelé événement impossible, alors que le deuxième est appelé événement certain.
%				\index{Événement impossible}\index{Événement certain}
%			\end{enumerate}%
%		\end{remark}%
%	\end{mydef}%
%	\begin{mydef}
%		Si un résultat de l'expérience est contenu dans $E$, on dit que $E$ est réalisé.
%	\end{mydef}%
%	\begin{mydef}
%		Soit $E$ et $F$ des événements d'un ensemble fondamental $S$.
%		\begin{enumerate}
%			\item $E\cup F$ dit l'union de $E$ et $F$, est l'événement qui est réalisé si 
%		\end{enumerate}
%	\end{mydef}%
\section[Chapitre 2-3]{Théorème de probabilité et probabilités conditionnelles}
\begin{enumerate}
	\item Si $\emptyset$ est l'ensemble vide, alors $P(\emptyset)=0$
	\item Si $S$ est l'espace échantillonnal, alors $P(S)=1$
	\item Si $E$ et $F$ sont deux événements, alors
	\[P(E\cup F) = P(E) + P(F) - P(E\cap F)\]
	\item Si $E$ et $F$ sont des événements mutuellement exclusifs, alors
	\[P(E\cup F) = P(E) + P(F)\]
	\item Si $E$ et $E^c$ sont des événements complémentaires, alors
	\[P(E) = 1 - P(E^c)\]
	\item La probabilité conditionnelle de l'événement $E$ sachant l'événement $F$ dénoté par $P(E|F)$ et est définie par
	\[P(E|F)=\frac{P(E\cap F)}{P(F)}\]
	\item Deux événements $E$ et $F$ sont dits indépendant si et seulement si
	\[P(E\cap F)=P(E)\cdot P(F)\]
	$E$ est dit statistiquement indépendant de $F$ si $P(E|F)=P(E)$ et $P(F|E)=P(F)$
	\item Les événements $E_1,E_2,...,E_n$ sont appelés mutuellement indépendant pour toute combinaison si et seulement si chaque combinaison de ces événements pris à un nombre quelconque à la fois est indépendante.
	\item (Théorème de Bayes) Si $E_1,E_2,...,E_n$ sont $n$ événements mutuellement exclusifs où leur union est l'espace échantillonnal $S$, et $E$ est un événement arbitraire de $S$ tel que $P(E)\neq 0$, alors
	\[P(E_k|E)=\frac{P(E|E_k)\cdot P(E_k)}{\sum_{j=1}^{n}P(E|E_j)\cdot P(E_j)}\]
	\item Si $E$ et $F$ sont des événements indépendants, alors $E$ et $F^c$ le sont aussi	
\end{enumerate}