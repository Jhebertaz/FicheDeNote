%prop:
%def : definition
%eq  : equation
%id	 : identity
\section[Chapitre 2]{Axiomes de probabilités}
	\begin{mydef}[Ensemble fondamental]\index{Ensemble fondamental}
		L'ensemble des résultats possibles d'une expérience est appelé ensemble fondamental et est noté $S$.
	\end{mydef}%
	\begin{mydef}[Événement]\index{Événement}
		Tout sous-ensemble $E$ de $S$ est appelé événement.
		\begin{remark}~
			\begin{enumerate}
				\item L'événement $\{a\}$ contenant un seul élément de $S$ est appelé événement élémentaire.\index{Événement élémentaire}
				\item L'ensemble vide noté $\emptyset$ et $S$ sont des événements. Le premier est appelé événement impossible, alors que le deuxième est appelé événement certain.
				\index{Événement impossible}\index{Événement certain}
			\end{enumerate}%
		\end{remark}%
	\end{mydef}%
	\begin{mydef}
		Si un résultat de l'expérience est contenu dans $E$, on dit que $E$ est \textbf{réalisé}.
	\end{mydef}%
	\begin{mydef}
		Soit $E$ et $F$ des événements d'un ensemble fondamental $S$.
		\begin{enumerate}\index{Opérations sur les ensembles}
			\item $E\cup F$ dit l'\textbf{union} de $E$ et $F$, est l'événement qui est réalisé si $E$ \underline{ou} $F$ est réalisé.
			\item $E \cap F$, appelé l'\textbf{intersection} de $E$ et $F$, est l'événement qui est réalisé si $E$ \underline{et} $F$ sont tous les deux réalisés.
			\item $E^c$ appelé le \textbf{complémentaire} de $E$ dans $S$, est l'événement qui est réalisé si $E$ n'est pas réalisé.
		\end{enumerate}%
	\end{mydef}%
	\begin{mydef}[\'Evénement mutuellement exclusifs]\index{\'Evénement mutuellement exclusifs}
		Si $E\cap F = \emptyset$, $E$ et $F$ sont dits mutuellement exclusifs (ou disjoints ou incompatibles).
	\end{mydef}%
	\begin{mythm}
		\begin{IEEEeqnarray*}{rCl}
			P(E_1\cup E_2\cup\cdots\cup E_n)=P\left(\bigcup_{i=1}^{n}E_i\right)=
		\end{IEEEeqnarray*}
	\end{mythm}
	
