\section*{Question 3}
	Soit $f:[0,2\pi]\longrightarrow\mathbb{R}$ une fonction continue telle que $f(0)=f(2\pi)$. Montrer qu'il existe un nombre réel $c\in [0,\pi]$ tel que $f(c)=f(c+\pi)$
\subsection*{Solution}
	Considérons la fonction $h:[0,\pi]\longrightarrow\mathbb{R}$ la fonction continue définie par $h(x)=f(x)-f(x+\pi)$ telle que
	\[\begin{cases}
		h(0)=f(0)-f(\pi)=f(2\pi)-f(\pi)\\
		h(\pi)=f(\pi)-f(2\pi)=f(\pi)-f(0)\\
	\end{cases}\]
	Par trichotomie, $\begin{cases}
	f(0)-f(\pi)<0\qquad (1)\\
	f(0)-f(\pi)=0\qquad (2)\\
	f(0)-f(\pi)>0\qquad (3)
	\end{cases}$, le cas (2) est évident ($c=0$ ou $c=\pi$). Pour le cas (1), on a que $h(0)=f(0)-f(\pi)<0\implies h(\pi)=-h(0)>0$. Donc, par le théorème des valeurs intermédiaire, 
	\[\exists c\in (0,\pi) : h(c)=f(c)-f(c+\pi)=0\implies f(c)=f(c+\pi)\] 
	Similairement pour le cas (2). Ce qui conclut la preuve.