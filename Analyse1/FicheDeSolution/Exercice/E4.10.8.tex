	\section*{4.10.8}
		\subsection{\normalfont{Soit $f:[0,2\pi]\longrightarrow\mathbb{R}$ une fonction continue telle que $f(0)=f(2\pi)$. Montrer qu'il existe un nombre $c\in[0,2\pi]$ tel que $f(c)=f(c+\pi)$.}}
		\subsubsection*{Solution}
		Considérons $h:[0,\pi]\longrightarrow\mathbb{R}$ une fonction continue (thm. \textit{algèbre de fonction continue}) définie par $h(x)=f(x)-f(x+\pi)$. Alors,
		\[
		\begin{cases}
		h(0)=f(0)-f(\pi)\\
		h(\pi)=f(\pi)-f(2\pi)=f(\pi)-f(0)
		\end{cases}\]
		Par trichotomie des nombres réels,
		\begin{align}
			h(0)>h(\pi)\\
			h(0)=h(\pi)\\
			h(0)<h(\pi)
		\end{align}
		Sachant que $h(0)+h(\pi)=0$, alors de (1) et (3),
		\[h(0)h(\pi)<0\]
		En particuliers, $h$ étant continue sur $[0,\pi]$ et respectant : $h(0)\neq h(\pi)$. Par le théorème des valeurs intermédiaires,
		\[\exists c\in[0,\pi] : 0=h(c)=f(c)-f(c+\pi) \]
		De (2), l'existence d'un $c\in [0,\pi]$ tel que $f(c)=f(c+\pi)$ est triviale, $c=0$ et/ou $c=\pi$.
%		\subsection{\normalfont{Montrer que tout polynôme de degré impair à coefficients réels possède au moins une racine.}}
%		\subsubsection*{Solution}
%		Par définition, un polynôme de degré impair est telle que $p(-x)=-p(x)$. Similairement aux situations (1), (2) et (3) du numéro précédant, on par la continuité des fonctions polynomiale (thm.) et par le théorème de valeurs intermédiaires :
%		\[\exists c \in\mathbb{R} : p(-c)+p(c)=0\]