	\section*{4.8.1.2}
		\subsection{\normalfont{Montrer que la fonction $f:(a,1)\longrightarrow\mathbb{R},0<a<1$ définie par $f(x)=1/x$, est uniformément continue sur $(a,1)$.}}
			Utilisons le fait que $f$ est continue en tout point sauf en $x=0$. Puisque $a>0$, alors $(a,1)\subset[a/2,1]=E$.
			
			Par un théorème toute fonction continue sur un compact est uniformément continue. Ainsi, $f$ est uniformément continue sur $E$ et par conséquent l'est aussi sur $(a,1)$.
		\subsection{\normalfont{Montrer que la fonction $f(x)=\frac{1}{1+x^2}$ est uniformément continue sur $(0,\infty)$.}}
			Par définition, $f$ est uniformément continue sur $E$ si,
			\[
				\forall\varepsilon>0,\exists\delta_{(\varepsilon)}>0 : \forall x,y\in E, |x-y|<\delta \implies \big|f(x)-f(y)\big|<\varepsilon
			\]
			\begin{align*}
			\intertext{Soit $x,y\in (0,\infty)$ et $0\leq|x-y|<\delta$,}
				\big|f(x)-f(y)\big|&=\bigg|\frac{1}{1+x^2}-\frac{1}{1+y^2}\bigg|\\
				&=\bigg|\frac{x^2-y^2}{(1+x^2)(1+y^2)}\bigg|\\
				&\leq\bigg|\frac{\delta (x+y)}{(1+x^2)(1+y^2)}\bigg|\\
				&\leq\delta\bigg|\frac{x}{(1+x^2)(1+y^2)}+\frac{y}{(1+x^2)(1+y^2)}\bigg|
			\intertext{Si $a\in (0,1), a^2<a<1\implies a<1+a^2$. Ainsi, $\frac{a}{1+a^2}<1$.}
			\intertext{Sinon, $a\in [1,\infty), 1 \leq a<a^2\implies a< a^2+1$. Ainsi, $\frac{a}{1+a^2}<1$}
			\intertext{En particuliers, $\forall a\in (0,\infty), \frac{a}{1+a^2}<1$. Alors,}
				\delta\bigg|\frac{x}{(1+x^2)(1+y^2)}+\frac{y}{(1+x^2)(1+y^2)}\bigg|&<2\delta
			\intertext{En posant, $\delta =\frac{\varepsilon}{2}$}
					\forall\varepsilon>0,\exists\delta_{(\varepsilon)}>0 : \forall x,y\in(0,\infty), |x-y|<\delta&\implies\big|f(x)-f(y)\big|<2\delta=\varepsilon
			\end{align*}
			Donc, la fonction $f$ est uniformément continue sur $(0,\infty)$.