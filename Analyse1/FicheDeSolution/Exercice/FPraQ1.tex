\section*{Question 1}
	Déterminer si les séries suivantes convergent et justifier.
	\begin{enumerate}[label=\alph*)]
		\item $\sum\limits_{n=1}^{+\infty}\frac{\sqrt{n}}{n^2-\pi}$
		\item $\sum\limits_{n=1}^{+\infty}\frac{2^nn!}{n^n}$
	\end{enumerate}
\subsection{Solution}
	Soit la série $\sum_{n=1}^{\infty}\frac{1}{\sqrt[2/3]{n}-4}$ qui converge par le critère de Riemann (la \textit{p-series} avec $3/2=p>1$).
	Par comparaison,
	\[\frac{\sqrt{n}}{n^2-\pi}\leq \frac{1}{\sqrt[2/3]{n}-4} \implies \sum_{n=1}^{\infty}\frac{\sqrt{n}}{n^2-\pi}\leq \sum_{n=1}^{\infty}\frac{1}{\sqrt[2/3]{n}-4} \]
	la série  $\sum\limits_{n=1}^{+\infty}\frac{\sqrt{n}}{n^2-\pi}$ converge.
\subsection{Solution}
	Par le critère de D'Alembert,
	\begin{align*}
		\lim_{n \to \infty}\Big|\frac{2^{n+1}(n+1)!}{(n+1)^{n+1}}\cdot\frac{n^n}{2^n n!}\Big|&=\lim_{n \to \infty}\Big|\frac{2n^n}{(n+1)^n}\Big|\\
		&=2\lim_{n \to \infty}\Big|\frac{n}{(n+1)}\Big|^n  \longrightarrow \frac{2}{e}<1
	\end{align*}
	la série converge absolument.