\section*{4.8.14.}
	Une fonction $f:I\longrightarrow\mathbb{R}$ qui est définie sur un intervalle $I\subseteq\mathbb{R}$ est dit \textit{lipschitzienne} s'il existe un nomber réel $K>0$ tel que pour tous $x,y\in I$, on a\[|f(x)-f(y)|\leq K|x-y|.\]
	Montrer qu'une telle fonction est uniformément continue sur $I$, Toutes les fonctions uniformément continues sur $I$ sont-elles lipschitziennes ?
	\subsection*{Solution}
	Puisque $K>0$ est un nombre fixe et $|x-y|<\delta$, alors
	\[|f(x)-f(y)|\leq K\delta\]
	Posons $\varepsilon=K\delta$ tel que
	\[\forall \varepsilon>0,\exists\delta>0 : \forall x,y\in I, |x-y|<\delta \implies |f(x)-f(y)|<\varepsilon\]
	La réciproque est généralement fausse (\textit{e.g} $f(x):=\sqrt{x}$)