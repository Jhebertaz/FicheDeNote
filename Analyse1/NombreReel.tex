\section{Les nombre réels}
	\begin{mythm}\index{Trichotomie}
		Les nombres réels sont ordonné tel que 
		\[\forall a,b \inreal_{\geq 0}, a+b \geq 0\]
		\[a \inreal, \begin{cases}
		a<0\\
		a=0\\
		a>0
		\end{cases}\]
	\end{mythm}
	\begin{mythm}\index{Axiome de complétude}
		Soit $\mathbb{R}\supset E\neq\emptyset$,\\
		$E$ borné supérieurement (resp. inférieurement) possède un \textit{supremum} (resp. \textit{infimum}) dans $\mathbb{R}$
	\end{mythm}
	\begin{myprop}\index{Archimède}
		Soit $x,y\inreal, x>0, x<y \implies \exists n\ininteger$ tel que $nx>y$
	\end{myprop}
	\begin{mydef}\index{Valeur absolue}\index{Inégalité triangulaire}
		\[x\inreal,|x|\leq b\Longleftrightarrow -b\leq x\leq b\]
		\[x,y\inreal, |x\cdot y|= |x|\cdot |y|\]
		\[\forall x,y\inreal,|x\pm y|\leq |x|+|y|\]
		\[\forall x,y\inreal, \big||x|-|y|\big|\leq |x\pm y|\]
	\end{mydef}