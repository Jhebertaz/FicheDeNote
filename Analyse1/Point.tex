\section{Les points}
	\begin{mydef}\index{Point intérieur}
		Un point $a\in E \subset\mathbb{R}$ est un point intérieur de $E$ si \[\exists\delta_{>0} : V(a,\delta)\subset E\]
	\end{mydef}
	\begin{mydef}\index{Point d'accumulation}
			Un point $a\inreal$ est un point d'accumulation de $E\subset\mathbb{R}$ si \[\forall\delta_{>0} : V'(a,\delta)\cap E\neq\emptyset\]
			Remarque : $a\notin E \nRightarrow a\notin E'$
	\end{mydef}
	\begin{mydef}\index{Point adhérent}
		Un point $a\inreal$ est un point adhérent de $E\subset\mathbb{R}$ si, \[\forall\delta_{>0}, V(a,\delta)\cap E\neq\emptyset\]
		Remarque:\\
		 \[a\in \bar{E}\implies a\in E'\]
		 \[a\in E\implies a\in\bar{E}\]
	\end{mydef}