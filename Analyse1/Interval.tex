\section{Les intervalles}
	\begin{mydef}\index{Intervalle}
		$I$ est un intervalle $\subset\mathbb{R}$ si lorsque $x,y\in I : x<y\implies \forall z\inreal : x<z<y$ est dans $I$
	\end{mydef}
	\begin{mydef}\index{Borné}
		$I$ est borné s'il possède un $\sup I =b$ et un $\inf I =a$ où $a,b\inreal$
	\end{mydef}
	\begin{mydef}\index{Non-borné}
		~
		\begin{itemize}
			\item Non-borné sup. : $\sup I \notin\mathbb{R}$
			\item Non-borné inf. : $\inf I \notin\mathbb{R}$
			\item Non-borné  : 
		\end{itemize}
	\end{mydef}
	\begin{mydef}\index{Voisinage}
		Voisinage centré en $x\inreal$ de rayon $\delta>0$ : $V(x,\delta)$ est l'intervalle ouvert \[(x-\delta,x+\delta)\]
	\end{mydef}
	\begin{mydef}\index{Vosinage pointé}
		Voisinage pointé... : $V'(x,\delta)=V(x,\delta)\backslash\{x\}$
	\end{mydef}