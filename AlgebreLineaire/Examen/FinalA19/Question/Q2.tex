\section*{Question 2}
\subsection{}
Pour trouver la représentation de la base $[T]_B$,
\[\{T(b_1),T(b_2),T(b_3),T(b_4)\}\]
\[
\begin{Bmatrix}
\left(
\begin{array}{cc}
-1 & 3 \\
1 & 2 \\
\end{array}
\right), & \left(
\begin{array}{cc}
1 & 2 \\
1 & 2 \\
\end{array}
\right), & \left(
\begin{array}{cc}
0 & 0 \\
-1 & 3 \\
\end{array}
\right), & \left(
\begin{array}{cc}
0 & 0 \\
1 & 2 \\
\end{array}
\right) \\
\end{Bmatrix}
\]
Autrement dit,
\[\begin{cases}
	T(b_1)=(-1,3,0,0)\bullet B\\
	T(b_2)=(1,2,0,0)\bullet B\\
	T(b_3)=(0,0,-1,3)\bullet B\\
	T(b_4)=(0,0,1,2)\bullet B
\end{cases}\implies [T]_B=
\begin{pmatrix}
 -1 & 1 & 0 & 0 \\
3 & 2 & 0 & 0 \\
0 & 0 & -1 & 1 \\
0 & 0 & 3 & 2
\end{pmatrix}
\]
\subsection{}
Réecrivons les bases matricielles en bases vectorielles, et ce en considérant 
\[
\begin{pmatrix}
a_{11} & a_{12} \\
a_{21} & a_{22} \\
\end{pmatrix}
\longrightarrow (a_{11},a_{12},a_{21},a_{22})^T\]
Ainsi, la bases $B$ et $B'$ deviennent respectivement,
\[B:=
\begin{pmatrix}
1 & 0 & 0 & 0 \\
0 & 1 & 0 & 0 \\
0 & 0 & 1 & 0 \\
0 & 0 & 0 & 1 \\
\end{pmatrix},\quad B':=
\begin{pmatrix}
1 & 0 & 0 & 0 \\
0 & 1 & 1 & 0 \\
0 & 1 & -1 & 0 \\
0 & 0 & 0 & 1 \\
\end{pmatrix}
\]

D'ailleurs sous cette forme, la base $B'$ est la matrice de passage $P_{B'\leftarrow B}$.
\subsection{}
De même qu'en a), pour la $B'$ 
\[\{T(b'_1),T(b'_2),T(b'_3),T(b'_4)\}\]
\[
\begin{Bmatrix}
\left(
\begin{array}{cc}
-1 & 3 \\
0 & 0 \\
\end{array}
\right), & \left(
\begin{array}{cc}
1 & 2 \\
0 & 0 \\
\end{array}
\right), & \left(
\begin{array}{cc}
0 & 0 \\
-1 & 3 \\
\end{array}
\right), & \left(
\begin{array}{cc}
0 & 0 \\
1 & 2 \\
\end{array}
\right) \\
\end{Bmatrix}
\]
Autrement dit,
\[\begin{cases}
T(b'_1)=\Big(-1,\frac{3}{2},\frac{3}{2},0\Big)\bullet B'\\
T(b'_2)=\Big(1,\frac{1}{2},\frac{3}{2},3\Big)\bullet B'\\
T(b'_3)=\Big(1,\frac{3}{2},\frac{1}{2},-3\Big)\bullet B'\\
T(b'_4)=\Big(0,\frac{1}{2},-\frac{1}{2},2\Big)\bullet B'
\end{cases}\implies [T]_{B'}=
\begin{pmatrix}
-1 & 1 & 1 & 0 \\
\frac{3}{2} & \frac{1}{2} & \frac{3}{2} & \frac{1}{2} \\
\frac{3}{2} & \frac{3}{2} & \frac{1}{2} & -\frac{1}{2} \\
0 & 3 & -3 & 2 \\
\end{pmatrix}
\]
\subsection{}
La relation est la suivante;
 $[T]_{B'}=P_{B'\leftarrow B}[T]_B P_{B\leftarrow B'}$ 
