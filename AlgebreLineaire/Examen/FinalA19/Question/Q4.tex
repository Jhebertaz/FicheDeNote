\section*{Question 4}
\subsection{}
Le polynôme caratéristique de la matric $A$ est :
\[\begin{split} P_A(\lambda)&=
\begin{vmatrix}
4-\lambda  & 0 & 0 & 0 \\
0 & 2-\lambda  & 0 & 2 \\
0 & 0 & 4-\lambda  & 0 \\
0 & 2 & 0 & 2-\lambda  \\
\end{vmatrix}\\
&=\lambda ^4-12 \lambda ^3+48 \lambda ^2-64 \lambda\\
&=(\lambda -4)^3 \lambda
\end{split}\]
On conclut que $\lambda = 4$ est une racine du polynôme caratéristique de $A$, donc une de ses valeurs propres. Elle est d'ailleurs de multiplicité algébrique de 3.
\subsection{}
Selon le résultat obtenue ci-dessus, les valeurs propres de $A$ sont $\lambda_1=4$ et $\lambda_2=0$ respectivement de multiplicité algébrique de 3 et de 1. \\
Pour $\lambda_1$,
\[\begin{split}
\left(
\begin{array}{cccc}
4-\lambda _1 & 0 & 0 & 0 \\
0 & 2-\lambda _1 & 0 & 2 \\
0 & 0 & 4-\lambda _1 & 0 \\
0 & 2 & 0 & 2-\lambda _1 \\
\end{array}
\right)&=\left(
\begin{array}{cccc}
0 & 0 & 0 & 0 \\
0 & -2 & 0 & 2 \\
0 & 0 & 0 & 0 \\
0 & 2 & 0 & -2 \\
\end{array}
\right)\\
&\sim \left(
\begin{array}{cccc}
0 & 1 & 0 & -1 \\
0 & 0 & 0 & 0 \\
0 & 0 & 0 & 0 \\
0 & 0 & 0 & 0 \\
\end{array}
\right)
\end{split}\]
Donc,
$E_{\lambda_1}=Vect\left\{\begin{pmatrix}
1\\0\\0\\0
\end{pmatrix},\begin{pmatrix}
0\\1\\0\\1
\end{pmatrix},\begin{pmatrix}
0\\0\\0\\1
\end{pmatrix}\right\}$
\\
Pour $\lambda_2$,
\[\begin{split}
\left(
\begin{array}{cccc}
4-\lambda _2 & 0 & 0 & 0 \\
0 & 2-\lambda _2 & 0 & 2 \\
0 & 0 & 4-\lambda _2 & 0 \\
0 & 2 & 0 & 2-\lambda _2 \\
\end{array}
\right)&=\left(
\begin{array}{cccc}
4 & 0 & 0 & 0 \\
0 & 2 & 0 & 2 \\
0 & 0 & 4 & 0 \\
0 & 2 & 0 & 2 \\
\end{array}
\right)\\
&\sim\left(
\begin{array}{cccc}
1 & 0 & 0 & 0 \\
0 & 1 & 0 & 1 \\
0 & 0 & 1 & 0 \\
0 & 0 & 0 & 0 \\
\end{array}
\right)
\end{split}\]
Donc $E_{\lambda_2}=Vect\left\{\begin{pmatrix}
0\\-1\\0\\1
\end{pmatrix}\right\}$
\subsection{}
La matrice $P$ est :
\[\left(
\begin{array}{cccc}
0 & 0 & 1 & 0 \\
1 & 0 & 0 & -1 \\
0 & 1 & 0 & 0 \\
1 & 0 & 0 & 1 \\
\end{array}
\right)\]
En normalisant les vecteurs propres, on obtient la matrice $O$ et la matrice diagonale des valeurs propres $D$:
\[O=\left(
\begin{array}{cccc}
0 & 0 & 1 & 0 \\
\frac{1}{\sqrt{2}} & 0 & 0 & -\frac{1}{\sqrt{2}} \\
0 & 1 & 0 & 0 \\
\frac{1}{\sqrt{2}} & 0 & 0 & \frac{1}{\sqrt{2}} \\
\end{array}
\right),\qquad D=\left(
\begin{array}{cccc}
4 & 0 & 0 & 0 \\
0 & 4 & 0 & 0 \\
0 & 0 & 4 & 0 \\
0 & 0 & 0 & 0 \\
\end{array}
\right)\]
\subsection{}
