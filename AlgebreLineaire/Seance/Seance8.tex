\section{Changement de base et matrices associées à une transformation linéaire, rang}
	\subsection{Système de coordonnées, matrice de passage}
		\begin{mydef}
			\index{Matrice de passage}
			\index{Base canonique}
			Soit $\mathcal{V}=\mathbb{R}^n$ et $B=\{v_1,v_2,...,v_n\}$ une base de $\mathcal{V}$. La \textbf{matrice de passage} de la base $B$ à la \textbf{base canonique} est définie par $P_B=(v_1\:v_2\:...\: v_n)$, la matrice formée des vecteurs de $B$.
		\end{mydef}
		\begin{myprop}
			Tout vecteur $x\in\mathbb{R}^n$ (dans la base canonique) satisfait $x=P_B[x]_B\Longleftrightarrow [x]_B=P_B^{-1}x$.
		\end{myprop}
	\subsection{Système de coordonnées, matrice de passage et changement de base}
		\begin{mydef}
			\index{Matrice de changement de base}
			Soit $B=\{b_1, b_2, ...,b_n\}$ et $C=\{c_1, c_2, ...,c_n\}$ deux bases d'un espace vectoriel $\mathcal{V}$ de dimensions finie $n$. La \textbf{matrice de passage de la base} $B$ \textbf{à la base} $C$ est définie par $P_{C\leftarrow B}=\big([b_1]_C\: [b_2]_C\:...\:[b_n]_C\big)$. La matrice de changement de base de $B$ à $C$ est la matrice formée des coordonnées des vecteurs de $B$ dans la base $C$.
		\end{mydef}
		\begin{mythm}
			\index{Matrice de changement de base}
			La matrice de changement de base est unique et inversible telle que $\big(P_{C\leftarrow B}\big)^{-1}=P_{B\leftarrow C}$ et que pour tout vecteur $x\in\mathcal{V},[x]_C=P_{C\leftarrow B} [x]_B$.
			\\
			Si $P_B$ (resp. $P_C$) est la matrice de passage de $B$ (resp. $C$) à la base canonique alors, $P_{C\leftarrow B}=\big(P_C\big)^{-1}P_B$.
		\end{mythm}
	\subsection{Matrice associée à une transformation linéaire}
		\begin{mydef}
			\index{Matrice associée}
			\index{Transformation linéaire}
			\index{Base}
			Soit $T : \mathcal{V}\longrightarrow\mathcal{W}$ une transformation linéaire de $\mathcal{V}$ dans $\mathcal{W}$, telle que $\text{\normalfont{dim}}\ \mathcal{V}=n$ et $\text{\normalfont{dim}}\ \mathcal{W}=m$. Soit $B=\{b_1,b_2,...,b_n\}$ une base de $\mathcal{V}$ et $C=\{c_1,c_2,...,c_n\}$ une base de $\mathcal{W}$. La matrice associée à la transformation linéaire dans les bases $B$ et $C$ est donnée par
			\[[T]_{C\leftarrow B}=\Big(\big[T(b_1)\big]_C\: \big[T(b_2)\big]_C\: ...\: \big[T(b_n)\big]_C\Big)\in\mathbb{R}^{m\times n}\quad\text{et}\quad\big[T(v)\big]_C = [T]_{C\leftarrow B}[v]_B,\quad\forall v\in\mathcal{V}\]
		\end{mydef}
		\begin{myprop}
			\index{Espace d'arrivé}
			\index{Espace de départ}
			\index{Base}
			\index{Changement de base}
			\index{Matrice associée}
			Soit la transformation linéaire $T : \mathcal{V}\longrightarrow\mathcal{W}$ et soit deux bases différentes dans chacuns des espaces $\mathcal{V}$ et $\mathcal{W}$. Soit $B$ et $B'$ des bases de l'espace de départ et, $C$ et $C'$ deux bases de l'espace d'arrivé.
			\begin{itemize}
				\item $P_{B\leftarrow B'}$ (resp. $P_{C'\leftarrow C}$), la matrice de changement de base de $B'$ à $B$ (resp. $C$ à $C'$) dans $\mathcal{V}$ (resp. $\mathcal{W}$)
				\item $[T]_{C\leftarrow B}$ (resp. $[T]_{C'\leftarrow B'}$), la matrice associée à $T$ dans les bases $B$ et $C$ (resp. $B'$ et $C'$) 
			\end{itemize}
		Alors, \[[T]_{C'\leftarrow B'}=P_{C'\leftarrow C}[T]_{C\leftarrow B}P_{B\leftarrow B'}\]
		\end{myprop}
	\subsection{Rang d'une transformation linéaire et matrice associée}
		\begin{mydef}
			\index{Rang}
			\index{Dimension}
			\index{Matrice canonique}
			\index{Image}
			\index{Noyau}
			Soit $T:\mathcal{V}\longrightarrow\mathcal{W}$ une transformation linéaire telle que $\text{\normalfont{dim}}\ \mathcal{V}=n$ et $\text{\normalfont{dim}}\ \mathcal{W}=m$. Le \textbf{rang} de $T$ est égal à la dimension de son image : $\text{\normalfont{rang}}\ T=\text{\normalfont{dim}}(\text{\normalfont{im}}\ T)$.\\
			Si $A$ est la matrice canonique associée à $T:\mathbb{R}^n \longrightarrow\mathbb{R}^m$, alors $$\text{\normalfont{rang}}\ T=\text{\normalfont{rang}}\ A = \text{\normalfont{dim}}(\text{\normalfont{im}}\ A)$$ (nombre de pivot de $A$ et donc le nombre $r$ de variables de bases). Alors que $\text{\normalfont{dim}}(\ker A)=n-r$, le nombre de variables libres. Autrement dit, $$\text{\normalfont{dim}}(\text{\normalfont{im}}\ A) + \text{\normalfont{dim}}(\ker A)=n$$
		\end{mydef}
		\begin{mythm}
			\index{Transformation linéaire}
			\index{Rang}
			\index{Noyau}
			\index{Dimension}
			Soit $T : \mathcal{V}\longrightarrow\mathcal{W}$ une transformation linéaire, alors
			\[\text{\normalfont{rang}}\ T+\text{\normalfont{dim}}(\ker T)=\text{\normalfont{dim}}\ \mathcal{V}\quad\text{ou encore}\quad\text{\normalfont{dim}}(\text{\normalfont{im}}\ T)+\text{\normalfont{dim}}(\ker T)=\text{\normalfont{dim}}\ \mathcal{V}\]
		\end{mythm}
		\begin{myprop}
			\index{Isomorphisme}
			\index{Transformation linéaire}
			Si la transformation linéaire $T : \mathcal{V}\longrightarrow\mathcal{W}$ est un \textbf{isomorphisme} alors $\text{\normalfont{dim}}\ \mathcal{V}=\text{\normalfont{dim}}\ \mathcal{W}$.
		\end{myprop}

